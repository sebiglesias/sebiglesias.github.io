
\documentclass[letterpaper,11pt]{article}

\usepackage{latexsym}
\usepackage[empty]{fullpage}
\usepackage{titlesec}
\usepackage{marvosym}
\usepackage[usenames,dvipsnames]{color}
\usepackage{verbatim}
\usepackage{enumitem}
\usepackage[hidelinks]{hyperref}
\usepackage{fancyhdr}
\usepackage[english]{babel}
\usepackage{tabularx}
\input{glyphtounicode}


%----------FONT OPTIONS----------
% sans-serif
% \usepackage[sfdefault]{FiraSans}
\usepackage[sfdefault]{roboto}
% \usepackage[sfdefault]{noto-sans}
% \usepackage[default]{sourcesanspro}

% serif
% \usepackage{CormorantGaramond}
% \usepackage{charter}


\pagestyle{fancy}
\fancyhf{} % clear all header and footer fields
\fancyfoot{}
\renewcommand{\headrulewidth}{0pt}
\renewcommand{\footrulewidth}{0pt}

% Adjust margins
\addtolength{\oddsidemargin}{-0.5in}
\addtolength{\evensidemargin}{-0.5in}
\addtolength{\textwidth}{1in}
\addtolength{\topmargin}{-.5in}
\addtolength{\textheight}{1.0in}

\urlstyle{same}

\raggedbottom
\raggedright
\setlength{\tabcolsep}{0in}

% Sections formatting
\titleformat{\section}{
  \vspace{-4pt}\scshape\raggedright\large
}{}{0em}{}[\color{black}\titlerule \vspace{-5pt}]

% Ensure that generate pdf is machine readable/ATS parsable
\pdfgentounicode=1

%-------------------------
% Custom commands
\newcommand{\resumeItem}[1]{
  \item\small{
    {#1 \vspace{-2pt}}
  }
}

\newcommand{\resumeSubheading}[4]{
  \vspace{-2pt}\item
    \begin{tabular*}{0.97\textwidth}[t]{l@{\extracolsep{\fill}}r}
      \textbf{#1} & #2 \\
      \textit{\small#3} & \textit{\small #4} \\
    \end{tabular*}\vspace{-7pt}
}

\newcommand{\resumeSubSubheading}[2]{
    \item
    \begin{tabular*}{0.97\textwidth}{l@{\extracolsep{\fill}}r}
      \textit{\small#1} & \textit{\small #2} \\
    \end{tabular*}\vspace{-7pt}
}

\newcommand{\resumeProjectHeading}[2]{
    \item
    \begin{tabular*}{0.97\textwidth}{l@{\extracolsep{\fill}}r}
      \small#1 & #2 \\
    \end{tabular*}\vspace{-7pt}
}

\newcommand{\resumeSubItem}[1]{\resumeItem{#1}\vspace{-4pt}}

\renewcommand\labelitemii{$\vcenter{\hbox{\tiny$\bullet$}}$}

\newcommand{\resumeSubHeadingListStart}{\begin{itemize}[leftmargin=0.15in, label={}]}
\newcommand{\resumeSubHeadingListEnd}{\end{itemize}}
\newcommand{\resumeItemListStart}{\begin{itemize}}
\newcommand{\resumeItemListEnd}{\end{itemize}\vspace{-5pt}}

%-------------------------------------------
%%%%%%  RESUME STARTS HERE  %%%%%%%%%%%%%%%%%%%%%%%%%%%%


\begin{document}

%----------HEADING----------

\begin{center}
    \textbf{\Huge \scshape Sebastián Iglesias} \\ \vspace{1pt}
    \href{https://sebiglesias.com.ar}{Email: \underline{me (at) sebiglesias.com.ar}} $|$ 
    \href{https://linkedin.com/in/sebiglesias}{LinkedIn: \underline{linkedin.com/in/sebiglesias}} $|$
    \href{https://github.com/sebiglesias}{Github: \underline{github.com/sebiglesias}}
\end{center}

\section{Resumen Profesional}
    Soy un Ingeniero de Software experimentado con más de seis años de experiencia diseñando y desarrollando sistemas robustos y escalables en tecnologías cloud, backend y frontend. Con experiencia en programación funcional usando React, Node.js y TypeScript, he liderado proyectos de impacto que mejoraron los pipelines de CI/CD y las experiencias de los clientes, impulsando un crecimiento medible. Además, mi trabajo con sistemas basados en datos y microservicios usando Java, Spring y PostgreSQL ha perfeccionado mi capacidad para ofrecer soluciones innovadoras ante desafíos complejos. Busco oportunidades para aprovechar mis habilidades técnicas y de liderazgo para construir productos de software transformadores.

%-----------EXPERIENCE-----------
\section{Experiencia}
  \resumeSubHeadingListStart
    \resumeSubheading
      {Ingeniero de Software}{Jul 2022 -- Presente}
      {Caribou}{Rol Remoto}
      \resumeItemListStart
        \resumeItem{Diseñé, documenté, lideré y desarrollé numerosas funcionalidades que abarcan pipelines de CI/CD, servicios en la nube en GCP, tareas de frontend y backend utilizando React, Node.js, TypeScript y Express (enfatizando un enfoque de programación funcional). Establecí e implementé varios tipos de pruebas utilizando Playwright para asegurar la robustez del producto. Trabajé estrechamente con los fundadores y clientes para comprender mejor sus necesidades y luego definir, liderar y desarrollar junto a ellos características clave del producto, impulsando un crecimiento significativo en nuestra base de usuarios.}
      \resumeItemListEnd
    \resumeSubheading
      {Ayudante de Cátedra}{Feb 2023 -- Presente}
      {Universidad Austral}{Buenos Aires, Argentina}
      \resumeItemListStart
        \resumeItem{Preparar y dar classes, tareas y exámenes a estudiantes de ingeniería en informática.}
        \resumeItem{Introducción a la Programación 1 (estudiantes de primer año): Python básico y principios de programación, uso de una terminal estándar POSIX, git y Github.}
        \resumeItem{Laboratorio I (estudiantes de 3er año):
        Este curso proporciona el conocimiento fundamental necesario para desarrollar un proyecto de software, abarcando todas las etapas desde la concepción hasta la implementación y pruebas. Los estudiantes trabajan en grupos y proponen proyectos de software del mundo real, utilizando análisis de casos de uso para definir los requisitos y crear un cronograma de implementación. El curso evalúa el progreso a través de dos entregas parciales, donde cada grupo muestra la implementación de sus casos de uso.}
        \resumeItemListEnd
    \resumeSubheading
      {Ingeniero de Software Senior}{Dec 2017 -- Jul 2022}
      {MuleSoft, a Salesforce company}{Buenos Aires, Argentina}
      \resumeItemListStart
        \resumeItem{Ingeniero de Software Senior: Desarrollé Mulesoft RPA, adaptando una base de código heredada al estilo de codificación de Mulesoft. Utilicé microservicios en Java y Spring que interactuaban con una base de datos Postgresql con migraciones gestionadas por Liquibase, interactuando con diferentes departamentos de ingeniería. (Jul 2021 -- Jul 2022)}
        \resumeItem{Ingeniero de Software: Desarrollé el Datagraph de Anypoint en un equipo de ingeniería siguiendo metodologías ágiles, desarrollando servicios frontend y backend. En conjunto con mi equipo, desarrollamos Servicios REST utilizando Java y Spring, así como clientes web con React, Redux y Typescript. También desarrollé pruebas de integración utilizando Testcafe. (Feb 2020 -- Jul 2021}
        \resumeItem{Ingeniero de Software Asociado: Desarrollé la Plataforma de Monitoreo dentro de la plataforma Anypoint de Mulesoft en un equipo de ingeniería siguiendo metodologías ágiles, desarrollando servicios frontend y backend. En conjunto con mi equipo, desarrollamos Servicios REST utilizando Scala, Akka y Java, y varios tipos de bases de datos, así como clientes web con React, Angular y Typescript. Todo ello fue desplegado en la nube de AWS mediante herramientas como Terraform, salt, spinnaker y Jenkins. Para fines de monitoreo, trabajé con el stack de elastic search y configuré scripts en Python y bash.}
    \resumeItemListEnd

    \resumeSubheading
      {Profesor de IT}{Apr 2016 -- Dec 2016}
      {FINES}{Buenos Aires, Argentina}
      \resumeItemListStart
        \resumeItem{Trabajé como profesor de informática en un curso acelerado de secundaria para adultos bajo el plan FINES. Planifiqué y di clases nocturnas tanto a adultos como a adolescentes.}
      \resumeItemListEnd

    \resumeSubheading
      {Voluntariado}{Mar 2015 -- May 2015}
      {Entrepreneur 4 Entrepreneur}{Buenos Aires, Argentina}
      \resumeItemListStart
        \resumeItem{Fui voluntario en el área de consultoría e informática. Estudié y apliqué metodologías LEAN startup para ayudar a otros emprendedores, y también realicé desarrollo web para la empresa de consultoría.}
      \resumeItemListEnd
    
    \resumeSubheading
      {IT Manager}{Aug 2014 -- Feb 2015}
      {I-Gen Latam}{Buenos Aires, Argentina}
      \resumeItemListStart
        \resumeItem{Realicé reparación de computadoras, gestión de bases de datos MySQL, desarrollo web con PHP y mantenimiento de un sistema legado de generación de informes.}
      \resumeItemListEnd
  \resumeSubHeadingListEnd


%-----------PROJECTS-----------
\section{Projects}
    \resumeSubHeadingListStart
      \resumeProjectHeading
          {\textbf{Informática Solidaria} $|$ \emph{NGO organization}}{Sep 2015 -- Sep 2017}
          \resumeItemListStart
            \resumeItem{Fundé Informática Solidaria, una organización que recibía equipos electrónicos usados, los reparaba y los donaba a instituciones educativas en Pilar, Buenos Aires, Argentina.}
          \resumeItemListEnd
    \resumeSubHeadingListEnd

%-----------EDUCATION-----------
\section{Educación}
  \resumeSubHeadingListStart
    \resumeSubheading
      {ITBA}{Buenos Aires, Argentina}
      {Curso de Posgrado, Especialización en Data Science}{Aug. 2019 -- Aug. 2020}
    \resumeSubheading
      {Universidad Austral}{Buenos Aires, Argentina}
      {Ingeniería en Informática}{Feb. 2014 -- Dec. 2018}
  \resumeSubHeadingListEnd

%-----------Skills-----------

\section{Technical Skills}
 \begin{itemize}[leftmargin=0.15in, label={}]
    \small{\item{
     \textbf{Lenguajes}{: Java, Python, SQL (Postgres), JavaScript, HTML/CSS, R, Typescript, Scala, Bash} \\
     \textbf{Frameworks/librerias}{: React, Node.js, FastAPI, JUnit, Spring, Express, PostHog} \\
     \textbf{Herramientas de Desarrollador}{: Git, GitHub, Docker, Jenkins, CircleCI, Google Cloud Platform (GCP), AWS, Intellij, Kubernetes, Nginx, Redis} \\
    }}
 \end{itemize}

%-------------------------------------------
\end{document}
